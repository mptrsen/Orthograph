\documentclass[a4paper]{scrartcl}
\usepackage[T1]{fontenc}			
\usepackage[utf8]{inputenc}	
\usepackage{lmodern}			
\usepackage{microtype}	

\title{Orthograph}
\author{Malte Petersen}
\date{\today}

\begin{document}
\maketitle
\tableofcontents

\section{Notes on HMMer3}

HMMer3 uses three notions to report endpoints of the alignment regions: Two
give the endpoints of the reported local alignment with respect to the query
model (``hmm from'' and ``hmm to'') and the target sequence (``ali from'' and
``ali to''). The third defines the \emph{envelope} (``env from'' and ``env to'')
of the domain's location on the target sequence. It is mostly a little (or a
lot) wider than than what HMMer thinks is a reasonably confident alignment and
represents a subsequence whose endpoints are only fuzzily inferrable.

Orthograph uses the envelope coordinates for its analysis, and this is normally
fine when working with high-scoring domains that are not close to each other.
However, be aware that the envelope coordinates of hmmsearch hits can and will
overlap when two weaker-scoring domains are close to each other, and this may
lead to some hits getting excluded because overlaps are not allowed in ortholog
regions. 

\end{document}


